\documentclass{article}
\setlength{\parindent}{0pt}
\setlength{\parskip}{2ex plus 0.5ex minus 0.2ex}
\usepackage[margin=1in]{geometry}
\usepackage{graphicx}
\usepackage{hyperref}
\usepackage{cleveref}

\graphicspath{{./figures/}}

\begin{document}
\section{Comprehensive, expensive (3D) models}
\subsection{MSBR}

The most detailed report on ORNL's molten salt breeder reactor is the technical
report by Robertson. \cite{robertson1971conceptual} A nice overview of the MSBR
geometry is shown in \cref{fig:vertical,fig:horizontal,fig:zoom_horiz}.

\begin{figure}[htpb]
  \centering
  \includegraphics{vertical_MSBR_cross_section.png}
  \caption{Vertical cross section of MSBR.}
  \label{fig:vertical}
\end{figure}
\begin{figure}[htpb]
  \centering
  \includegraphics{horizontal_MSBR_cross_section.png}
  \caption{Horizontal cross section of MSBR.}
  \label{fig:horizontal}
\end{figure}
\begin{figure}[htpb]
  \centering
  \includegraphics{zoomed_horizontal_MSBR_cross_section.png}
  \caption{Zoomed in horizontal cross-section of MSBR. Shows presence of
    interstitials between graphite block elements in zone 1 of the reactor.}
  \label{fig:zoom_horiz}
\end{figure}

\section{Simplified models}
\subsection{MSBR children}
\subsubsection{Cammi et. al.}
This model is based on the multi-physics model by \cite{cammi2011multi}.

\bibliographystyle{unsrt}
\bibliography{MSR-design}
\end{document}
